\documentclass[10pt,a4paper]{article}
\usepackage[utf8]{inputenc}
\usepackage[english]{babel}
\usepackage[T1]{fontenc}
\usepackage{amsmath}
\usepackage{amsfonts}
\usepackage{amssymb}
\usepackage{makeidx}
\usepackage{graphicx}
\usepackage{fourier}
\usepackage{listings}
\usepackage{color}
\usepackage{hyperref}
\usepackage[left=2cm,right=2cm,top=2cm,bottom=2cm]{geometry}
\author{Johannes Scheller, Vincent Noculak, Lukas Powalla}
\title{Computational Physics - Project 3}

\lstset{language=C++,
	keywordstyle=\bfseries\color{blue},
	commentstyle=\itshape\color{red},
	stringstyle=\color{green},
	identifierstyle=\bfseries,
	frame=single}
\begin{document}


\section{Theory}
\subsection{Ferromagnetic order}

A ferromagnet have a spontaneous magnetic moment even with the absence of an external magnetic field. Due the existence of a spontaneous moment the electron spin and magnetic moments must be arranged in a regular manner.
ferromagnet all spin aligned, antiferromagnet all spin align with neighboring pointing in opposite directions, ferrimagnet the opposing moments are unequal , etc.
   

\subsection{Periodic boundary conditions} 
Periodic boundary conditions is used for approximating a large or infinite system by using smaller repeating system, we will impose PBCs on our spin lattice in x and y directions.

 s(L+1,y) = s(1,y)
 
 s(x,L+1) = s(X,1)  

\subsection{Ising model}
Ising model is a mathematical model for ferromagnetism studies of phase transitions for magnetic system at given temperature, the model consists the interaction between two neighboring spins is related by the interaction energy 
\begin{equation}
  -Js_ks_l
\end{equation} 
where the sin s can be in two states +1 or -1,where  $s_k$ and $s_l$ are the nearest neighbors. Which give a low energy (-J) if the two spin aligned and high energy (j) for spin pointing in opposite direction. The total energy to a system with N number of spins and with the absence of magnetic field can be expressed as 
\begin{equation}
  E=-J\sum_{<kl>}^{N}s_ks_l
\end{equation}
...probability distribution with expectation value <E> ...
  
\subsection{Metropolis algorithm}

\subsection{critical temperature (Lars Onsager)}

\end{document}