\documentclass[10pt,a4paper]{article}
\usepackage[utf8]{inputenc}
\usepackage[english]{babel}
\usepackage[T1]{fontenc}
\usepackage{amsmath}
\usepackage{amsfonts}
\usepackage{amssymb}
\usepackage{makeidx}
\usepackage{graphicx}
\usepackage{fourier}
\usepackage{listings}
\usepackage{color}
\usepackage{hyperref}
\usepackage[left=2cm,right=2cm,top=2cm,bottom=2cm]{geometry}
\author{Johannes Scheller, Vincent Noculak, Lukas Powalla, Richard Asbah}
\title{Computational Physics - Project 4}

\lstset{language=C++,
	keywordstyle=\bfseries\color{blue},
	commentstyle=\itshape\color{red},
	stringstyle=\color{green},
	identifierstyle=\bfseries,
	frame=single}
\begin{document}

\maketitle
\newpage
\tableofcontents
\newpage

\section*{Introduction}
In project 4 we are dealing with the Ising model in two dimensions without an external magnetic field. We are looking at a lattice of L times L particles, which have spinvalues $\pm 1$. In order to compute different interesting values, we want to use the metropolis algorithm.  With our computations, we want to calculate the Energy, the absolute value of the magnetisation, the heat capacity and susceptibility of the system as a function of time. We also want to compare our solutions with the theoretical closed solution. 
This project may also show the link from statistical physics to macroscopic properties of a given physical system, which is a very interesting relation. 
\section{Theory}

\subsection{General properties of physical systems and their link to statistical physics}

\subsubsection{physical ensembles}

Canonical ensemble is a statistical way to represents the possible states in a system with fixed temperature, whereas the system exchange energy, the energy follows as an expectation value. The probability distribution is given by the Boltzmann distribution.

\begin{align}
P_i (\beta) =\frac{ e^{- \beta E_i}}{Z}
\end{align} 
$\beta = 1/k_B T$ where T is the temperature, $k_B$ is the Bolztmann constant,$E_i$ is the energy of microstate i and Z is the partition function for the canonical ensemble is the sum over all the microstates M.

\begin{equation}
  Z=\sum_{i=1}^{M}e^{- \beta E_i}
\end{equation}
   

\subsubsection{General properties of canonical ensembles}
the canonical ensemble pursuit towards an energy minimum and higher entropy expressed by Helmholtz' free energy.

\begin{equation}
F=-k_bTlnZ = <E>-TS \label{F}
\end{equation}
where the entropy S is given by
\begin{equation}
S=-k_blnZ + k_b T \frac{\partial lnZ}{\partial T}\label{S}
\end{equation}
after running the system for long time the canonical ensemble is uniquely determined and does not depend on the arbitrary choices for a given temperature, having a steady state without being affected by the equilibrium continuous motion.
the system uncertainty  due the Energy fluctuations in the canonical ensemble give the variance of the energy.\\
 
\centerline{ from equation \ref{F}, \ref{S} and probability distribution $P_i$ }
\begin{align}
<E> = k_b T^2 \frac{\partial lnZ}{\partial T}=\sum_{i=1}^{M} E_iP_i(\beta)=\frac{1}{Z }\sum_{i=1}^{M}E_i e ^{ - \beta E_i}
\end{align}
The heat capacity is how much the energy change due to the change in the temperature. The heat capacity $C_V$ can be defined as

\begin{equation}
C_V =\frac{\partial E}{\partial T} 
\end{equation}

\begin{equation}
\frac{\partial}{\partial T}\frac{1}{Z} = \frac{\partial}{\partial T}\frac{1}{\sum_{i=1}^{M}e^{- \beta E_i}} = \frac {E_i}{K_B T^2  \sum_{i=1}^{M}e^{- \beta E_i}} = \frac{E_i}{K_BT^2} \frac{1}{Z}
\end{equation}

\begin{equation}
\frac{\partial}{\partial T}\sum_{i=1}^{M}E_i e ^{ - \beta E_i} = -\frac{1}{k_BT^2}\sum_{i=1}^{M}E_i^2 e ^{ - \beta E_i}
\end{equation}

\begin{equation}
C_V =\frac{\partial <E>}{\partial T} = \frac{\partial}{\partial T} (\frac{1}{Z }\sum_{i=1}^{M}E_i e ^{ - \beta E_i}) = \frac{E_i}{Zk_BT^2}\sum_{i=1}^{M}E_i e ^{ - \beta E_i} -\frac{1}{Zk_BT^2}\sum_{i=1}^{M}E_i^2 e ^{ - \beta E_i} = \frac{1}{Zk_BT^2}<E^2>-<E>^2
\end{equation}

\subsubsection{Ferromagnetic order}

A ferromagnet have a spontaneous magnetic moment even with the absence of an external magnetic field. Due the existence of a spontaneous moment the electron spin and magnetic moments must be arranged in a regular manner.
ferromagnet all spin aligned, antiferromagnet all spin align with neighboring pointing in opposite directions, ferrimagnet the opposing moments are unequal , etc.

\subsection{theoretical numerical solutions}

\subsubsection{Ising model}
Ising model is a mathematical model for ferromagnetism studies of phase transitions for magnetic system at given a temperature. The model consists the interaction between two neighbouring spins is related by the interaction energy 
\begin{equation}
  -Js_ks_l
\end{equation} 
where the sin s can be in two states +1 or -1,where  $s_k$ and $s_l$ are the nearest neighbors. Which give a low energy (-J) if the two spin aligned and high energy (j) for spin pointing in opposite direction. The total energy to a system with N number of spins and with the absence of magnetic field can be expressed as 
\begin{equation}
  E=-J\sum_{<kl>}^{N}s_ks_l
\end{equation}
...probability distribution with expectation value <E> ...
  
  \subsubsection{Periodic boundary conditions} 
Periodic boundary conditions is used for approximating a large or infinite system by using smaller repeating system, we will impose PBCs on our spin lattice in x and y directions.

 s(L+1,y) = s(1,y)
 
 s(x,L+1) = s(X,1)  

\subsubsection{Metropolis algorithm}

\subsubsection{critical temperature (Lars Onsager)}

\subsection{Closed solution for a 2 dimensional 2 x 2 lattice}

We want now to look at a 2 x 2 lattice and we want to calculate the partition function, the energy, magnetisation, heat capacity and susceptibility of the system  dependent of T. 
The partition function for a canonical ensemble with periodic boundary conditions can be computed  by:
\begin{align}
Z= \sum_{i=1}^{M} e^{- \beta E_i}
\end{align} 
Here, $\beta$ is $\frac{1}{k_b \cdot T}$, where $k_b$ is the Bolzmann constant. 
In this expression we sum over all microstates m. The Energy of the system in configuration i is then:
\begin{align}
E_i = - J \sum_{<kl>}^N s_k s_l 
\end{align} 

The sum over $<kl>$ means that we only sum over nearest neighbours. In our 2 x 2 case, we have for each "particle" two possible values $\pm 1$. This means that we have all in all $2^{2 \cdot 2} = 2^4=16$ micro states. We have to compute the Energy of the micro states in order to compute the partition function. 
We also want to introduce the magnetisation, which is simply the sum over all the spins of the system:
\begin{align}
M_i=\sum_{j=1}^N s_j
\end{align}
We want also to introduce the so called degeneracy, which counts the number of micro states for a given micro energy. We get the following table:
\begin{figure}[h]
\centering
\caption{Energy of the different micro states}
\label{table of microstates}
\begin{tabular}{c|c|c|c}
Number of spins up (+1) & Degeneracy &  Energy & Magnetization\\
\hline \hline
4 & 1 & $-8J$ & 4 \\
3 & 4 & 0 & 2 \\
2 & 4 & 0 & 0 \\
2 & 2 & $8J$ & 0 \\
1 & 4 & 0 & -2 \\
0 & 1 & $-8J$ & -4 
\end{tabular}
\end{figure}
We can now write the expression of the partition function as in equation \ref{Partition 2x2}. We used the Table \ref{table of microstates} to calculate the sum over the micro states. 
\begin{align}
Z&= \sum_{i=1}^{M} e^{- \beta E_i}= 12 \cdot e^{-\beta \cdot 0 } + 2 \cdot e^{-8J \beta } + 1 \cdot e^{8J \beta } + 1 \cdot e^{8J \beta } \\
&= 12+ 2 \cdot e^{-8J \beta } + 2 \cdot e^{8J \beta } \\
&= 12+ 4 \cdot cosh \left( 8J \beta \right) \label{Partition 2x2}
\end{align} 
We can now calculate the expectation value of the energy. There are two possible ways of calculating it. the first way of calculating the expectation value of the energy can be seen in equation \ref{Energyexpectation way1}. 
\begin{align}
<E>&= - \frac{\partial ln(Z)}{\partial \beta} =\frac{1}{Z} \cdot 32J  \cdot sinh(8J \beta ) \\ \label{Energyexpectation way1}
&= \frac{32 J \cdot sinh((8J \beta )}{Z}\\
&=\frac{8 \cdot J \cdot  sinh(8J \beta ) }{3+cosh(8J\beta)}
\end{align}
Alternatively, we can calculate the expectation value of the Energy by looking at the micro states:
\begin{align}
<E> = \frac{1}{Z} \sum_{i=1}^{M} E_i e^{- \beta E_i}=\frac{8 \cdot J \cdot  sinh(8J \beta ) }{3+cosh(8J\beta)}
\end{align}
Both expressions are equal. Next, we want to determine the expectation value of the magnetisation. We use the formula \ref{expectation magnetisation 2x2}. We can see that we get 0 for the expectation value of the magnetisation. 
\begin{align}
<M> &= \frac{1}{Z} \sum_{i}^M M_i \cdot e^{- \beta E_i }\\\label{expectation magnetisation 2x2}
&= \frac{1}{Z} \cdot \left( 4 \cdot 1 \cdot e^{-8J\beta}+ 2 \cdot 4+(-2) \cdot 4 + (-4) \cdot 1 \cdot e^{-8J \beta } \right)\\
&=0
\end{align}
In order to describe how the temperature will change when thermal energy is added to the system, we want to look at a quantity called heap capacity. ($C_v$) The bigger this quantity is the less heats the system up by a given amount of thermal energy, which is added to the system.  
\begin{align}
C_v &= \frac{1}{k_b T^2} \left( \frac{1}{Z} \sum_{i=1}^{M} E_i^2 e^{- \beta E_i } - \left( \frac{1}{Z} \sum_{i=1}^{M} E_i e^{- \beta E_i }  \right)^2 \right)\\
&=\frac{1}{k_b T^2} \left( \frac{1}{Z} \left( 2 \cdot (8J)^2 \cdot e^{8J \beta}+2 \cdot (-8J)^2 \cdot e^{-8J \beta} \right) - \left( \frac{8 \cdot J \cdot  sinh(8J \beta ) }{3+cosh(8J\beta)} \right)^2 \right)\\
&=\frac{1}{k_b T^2} \left( \frac{64\cdot J\cdot cosh(8 J \beta )}{3+cosh(8J\beta)}  - \left( \frac{8 \cdot J \cdot sinh(8J \beta ) }{3+cosh(8J\beta)} \right)^2 \right)\\
&= \frac{1}{k_b T^2} \left( \frac{64 \cdot J +3\cdot J \cdot 64 cosh(8J \beta)}{\left(3+cosh(8J\beta)\right)^2} \right)\\
&= \frac{64}{k_b T^2} \left( \frac{ J +3 J \cdot cosh(8J \beta)}{\left(3+cosh(8J\beta)\right)^2} \right)\\
\end{align}
At last, we wan to have a look at the magnetic susceptibility. This quantity is a magnetic property of the material. The magnetic susceptibility describes the response of the material to an applied magnetic field. 
\begin{align}
\chi &= \frac{1}{k_b T} \cdot \left( \frac{1}{Z} \sum_{i=1}^{M} M_i^2 e^{- \beta E_i } - \left( \frac{1}{Z} \sum_{i=1}^{M} M_i e^{- \beta E_i }  \right)^2 \right)\\
&= \frac{1}{k_b T} \cdot \left( \frac{1}{Z} \cdot \left( 4^2 \cdot 1 \cdot e^{-8J\beta}+ 2^2 \cdot 4+(-2)^2 \cdot 4 + (-4)^2 \cdot 1 \cdot e^{-8J \beta } \right) - \left( 0  \right)^2 \right)\\
&= \frac{1}{k_b T} \cdot \frac{32 e^{-8J\beta}+32}{ 12+ 4 \cdot cosh \left( 8J \beta \right)}\\
&= \frac{1}{k_b T} \cdot \frac{8 e^{-8J\beta}+8}{ 3+ \cdot cosh \left( 8J \beta \right)}\\
\end{align}

\section{Execution}

\section{Comparison and discussion of results}

\section{source code}

\end{document}
