\documentclass[10pt,a4paper]{article}
\usepackage[utf8]{inputenc}
\usepackage[english]{babel}
\usepackage[T1]{fontenc}
\usepackage{amsmath}
\usepackage{amsfonts}
\usepackage{amssymb}
\usepackage{makeidx}
\usepackage{graphicx}
\usepackage{fourier}
\usepackage{listings}
\usepackage{color}
\usepackage{hyperref}
\usepackage[left=2cm,right=2cm,top=2cm,bottom=2cm]{geometry}
\author{Johannes Scheller, Vincent Noculak, Lukas Powalla, Richard Asbah}
\title{Computational Physics - Project 4}

\lstset{language=C++,
	keywordstyle=\bfseries\color{blue},
	commentstyle=\itshape\color{red},
	stringstyle=\color{green},
	identifierstyle=\bfseries,
	frame=single}
\begin{document}

\maketitle
\newpage
\tableofcontents
\newpage

\section*{Introduction}
In project 4 we are dealing with the Ising model in two dimensions without an external magnetic field. We are looking at a lattice of L times L particles, which have spinvalues $\pm 1$. In order to compute different interesting values, we want to use the metropolis algorithm.  With our computations, we want to calculate the Energy, the absolute value of the magnetisation, the heat capacity and susceptibility of the system as a function of time. We also want to compare our solutions with the theoretical closed solution. 
This project may also show the link from statistical physics to macroscopic properties of a given physical system, which is a very interesting relation. 
\section{Theory}

\subsection{General properties of physical systems and their link to statistical physics}

\subsubsection{physical ensembles}
% say something about canonical ensemble

\subsubsection{General properties of canonical ensembles}

%Part. funct. Energy and so on


\subsubsection{Ferromagnetic order}

A ferromagnet have a spontaneous magnetic moment even with the absence of an external magnetic field. Due the existence of a spontaneous moment the electron spin and magnetic moments must be arranged in a regular manner.
ferromagnet all spin aligned, antiferromagnet all spin align with neighboring pointing in opposite directions, ferrimagnet the opposing moments are unequal , etc.

\subsection{theoretical numerical solutions}

\subsubsection{Ising model}
Ising model is a mathematical model for ferromagnetism studies of phase transitions for magnetic system at given a temperature. The model consists the interaction between two neighbouring spins is related by the interaction energy 
\begin{equation}
  -Js_ks_l
\end{equation} 
where the sin s can be in two states +1 or -1,where  $s_k$ and $s_l$ are the nearest neighbors. Which give a low energy (-J) if the two spin aligned and high energy (j) for spin pointing in opposite direction. The total energy to a system with N number of spins and with the absence of magnetic field can be expressed as 
\begin{equation}
  E=-J\sum_{<kl>}^{N}s_ks_l
\end{equation}
...probability distribution with expectation value <E> ...
  
  \subsubsection{Periodic boundary conditions} 
Periodic boundary conditions is used for approximating a large or infinite system by using smaller repeating system, we will impose PBCs on our spin lattice in x and y directions.

 s(L+1,y) = s(1,y)
 
 s(x,L+1) = s(X,1)  

\subsubsection{Metropolis algorithm}

\subsubsection{critical temperature (Lars Onsager)}

\subsection{Closed solution for a 2 dimensional 2 x 2 lattice}

We want now to look at a 2 x 2 lattice and we want to calculate the partition function, the energy, magnetisation, heat capacity and susceptibility of the system  dependent of T. 
The partition function for a canonical ensemble with periodic boundary conditions can be computed  by:
\begin{align}
Z= \sum_{i=1}^{M} e^{- \beta E_i}
\end{align} 
Here, $\beta$ is $\frac{1}{k_b \cdot T}$, where $k_b$ is the Bolzmann constant. 
In this expression we sum over all microstates m. The Energy of the system in configuration i is then:
\begin{align}
E_i = - J \sum_{<kl>}^N s_k s_l 
\end{align} 

The sum over $<kl>$ means that we only sum over nearest neighbours. In our 2 x 2 case, we have for each "particle" two possible values $\pm 1$. This means that we have all in all $2^{2 \cdot 2} = 2^4=16$ micro states. We have to compute the Energy of the micro states in order to compute the partition function. 
We also want to introduce the magnetisation, which is simply the sum over all the spins of the system:
\begin{align}
M_i=\sum_{j=1}^N s_j
\end{align}
We want also to introduce the so called degeneracy, which counts the number of micro states for a given micro energy. We get the following table:
\begin{figure}[h]
\centering
\caption{Energy of the different micro states}
\label{table of microstates}
\begin{tabular}{c|c|c|c}
Number of spins up (+1) & Degeneracy &  Energy & Magnetization\\
\hline \hline
4 & 1 & $-8J$ & 4 \\
3 & 4 & 0 & 2 \\
2 & 4 & 0 & 0 \\
2 & 2 & $8J$ & 0 \\
1 & 4 & 0 & -2 \\
0 & 1 & $-8J$ & -4 
\end{tabular}
\end{figure}
We can now write the expression of the partition function:
\begin{align}
Z&= \sum_{i=1}^{M} e^{- \beta E_i}= 12 * e^{-\beta \cdot 0 } + 2 \cdot e^{-8J \beta } + 1 \cdot e^{9J \beta } + 1 \cdot e^{9J \beta } \\
&= 12+ 2 \cdot e^{-8J \beta } + 2 \cdot e^{9J \beta } \\
&= 12+ 4 \cdot cosh \left( 8J \beta \right)
\end{align} 


\section{Execution}

\section{Comparison and discussion of results}

\section{source code}

\end{document}
