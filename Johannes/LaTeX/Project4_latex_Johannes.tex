\documentclass[10pt,a4paper]{article}
\usepackage[utf8]{inputenc}
\usepackage[english]{babel}
\usepackage[T1]{fontenc}
\usepackage{amsmath}
\usepackage{amsfonts}
\usepackage{amssymb}
\usepackage{makeidx}
\usepackage{graphicx}
\usepackage{fourier}
\usepackage{listings}
\usepackage{color}
\usepackage{hyperref}
\usepackage[left=2cm,right=2cm,top=2cm,bottom=2cm]{geometry}
\author{Johannes Scheller, Vincent Noculak, Lukas Powalla, Richard Asbah}
\title{Computational Physics - Project 4}

\lstset{language=C++,
	keywordstyle=\bfseries\color{blue},
	commentstyle=\itshape\color{red},
	stringstyle=\color{green},
	identifierstyle=\bfseries,
	frame=single}
\begin{document}

\maketitle
\newpage
\tableofcontents
\newpage

\section*{Introduction}

\section{Theory}

\section{Execution}
\subsection{Implementing the Algorithm}
In all the calculations we did with the programm, we assume that 
Our program is split into three parts: in the first part, we prepare the calculations by declaring variables and initializing the grid. We call a function \emph{readINput} that reads input from the screen the desired temperature range, the temperature step size, the lattice size and preferences concerning the initial spin setup. Next, we call a function \emph{initialize} that initializes the array that contains all the spins. Depending on the user's choice, the lattice will be set up randomly or with all spins pointing in one direction. We also prepare the output file and the seed for the random number generator in the first part.

In the main part, we perform a loop over all desired temperatures. For every temperature, we 

\begin{lstlisting}
for (temp=tempStart; temp<=tempMax; temp+=tempStep){
	acceptedmoves=0;
//reset Energy and magnetization (averages)
for(int i=0; i<5; i++){
	averages[i]=0;
}
//Set array with possible energy changes according to temperature
for(int i=0; i<5; i++){
	double delEnergy = (4*i)-8;
	energyChanges[i] = exp(((double)-delEnergy)/((double)temp));
}
//thermalization - comment out for exercices where thermalization behaviour should be studied!
//thermalization(spinArray, size, idum, energyChanges, M, E, acceptedmoves);
//actual Monte Carlo happens here
for(int i=0; i<mccycles; i++){
	monteCarlo(spinArray, size, idum, energyChanges, M, E, acceptedmoves);
	averages[0]+=E;
	averages[1]+=E*E;
	averages[2]+=M;
	averages[3]+=M*M;
	averages[4]+=abs((int)M);
//Only for c), otherwise comment next line out (very slow!)
//output(size, i+1, temp, averages);

//only for c) (second part); comment out if not used!
//ofile << i << "\t" << acceptedmoves << endl;

//This is for part d, can be commented out else
ofile << E << endl;
}
//output of data for this temperature (comment out if not used!)
output(size, mccycles, temp, averages);
}
\end{lstlisting}

\subsection{Results}
As a first benchmark test for our program, we calculated the expectation values $<E>$, $<|M>|$, $<C_V>$ and $<\chi>$ of a $2\times2$-lattice for different temperatures. Those results could be easily compared to the analytical values from the part \ref{closed_solution}. In fig. \ref{b_E} -- \ref{b_chi}, you can see our results for these expectation values compared with the analytical solutions as functions of $T$. We took $10000$ Monte Carlo cycles for each temperature to achieve good results. You can see that the results fit very well to the analytical solutions which means that our program passed this benchmark test and works fine.

In the following table, we compared the analytical results for the different expectation values for a temperature $T=1.0$. It shows that all numerical results have a precision of at least two, in most cases of even three leading digits.
\begin{table}[h]
	\centering
	\caption{Analytical and numerical value of different expectaion values for $T=1.0$ in a $2\times2$-lattice}
	\begin{tabular}{ccc}
	& Analytical value & Numerical results \\\hline
	$<E>$ & $-7.983928$ & $-7.984080$ \\
	$<|M|>$ & $3.994642$ & $3.994760$ \\
	$<c_v>$ & $0.128329$ & $0.127107$ \\
	$<\chi>$ & $0.016043$ & $0.015494$	
	\end{tabular}	
\end{table}
\begin{figure}[h]
	\includegraphics[width=\textwidth]{Energy.png}
	\caption{$<E>$ in a $2\times 2$-lattice for different temperatures\label{b_E}}
\end{figure}
\begin{figure}[h]
	\includegraphics[width=\textwidth]{Magnetization.png}
	\caption{$<|M|>$ in a $2\times 2$-lattice for different temperatures\label{b_M}}
\end{figure}
\begin{figure}[h]
	\includegraphics[width=\textwidth]{Heat_capacity.png}
	\caption{$<c_v>$ in a $2\times 2$-lattice for different temperatures\label{b_Cv}}
\end{figure}
\begin{figure}[h]
	\includegraphics[width=\textwidth]{Susceptibility.png}
	\caption{$<\chi>$ in a $2\times 2$-lattice for different temperatures\label{b_chi}}
\end{figure}
In the next step, we took a closer look at the effect of thermalization. This term describes the process of the system slowly reaching the most likely state for a given temperature. When we start with a random setup, it is very unlikely that the system is already in this state at the beginning, but it will need some time (or, in our case, some Monte Carlo cycles) to reach it.

To get more insight in the process of thermalization, we observed the development of $<E>$ and $<|M|>$ of a $20\times20$-lattice for temperatures of $T=1.0$ and $T=2.4$, for both starting with a random setup and a lattice with all spins pointing in one direction. You can see that development in the figures \ref{c_1} -- \ref{c_4}, where these expectation values are plotted as function of the number of Monte Carlo cycles. We also plotted how many \glq moves\grq (flipping of spins) got accepted as a function of the number of cycles in fig. \ref{c_5} -- \ref{c_6} It is obvious that this value is proportional to the number of cycles and that, the higher the temperature is, the faster the number of accepted moves increases.

All these plots show that we need a bit more than $1000$ cycles in order to reach a stable state. This duration seems not to depend on the temperature, but obviously on the initial set-up: If this is close to a setup corresponding to the given temperature, it will not take long time to reach a stable state.

For later calculation, we set up a function called \emph{thermalization} for our program. This function performs the Metropolis algorithm without collecting data, instead it evaluates the energy after every 10 cycles. When the change between two measurements is less than $1\%$, the program starts collecting data. This takes in account the process described above and ensures that we reached a stable state before the actual calculations start.
\begin{figure}[h]
	\includegraphics[width=\textwidth]{a1o.png}
	\caption{$<E>$ and $<|M|>$ in a $20\times 20$-lattice as a function of the number of Monte Carlo cycles\label{b_chi}, starting with ordered spins at $T=1.0$}
\end{figure}
\begin{figure}[h]
	\includegraphics[width=\textwidth]{a1r.png}
	\caption{$<E>$ and $<|M|>$ in a $20\times 20$-lattice as a function of the number of Monte Carlo cycles\label{b_chi}, starting with random spins at $T=1.0$}
\end{figure}
\begin{figure}[h]
	\includegraphics[width=\textwidth]{a24o.png}
	\caption{$<E>$ and $<|M|>$ in a $20\times 20$-lattice as a function of the number of Monte Carlo cycles\label{b_chi}, starting with ordered spins at $T=2.4$}
\end{figure}
\begin{figure}[h]
	\includegraphics[width=\textwidth]{a24r.png}
	\caption{$<E>$ and $<|M|>$ in a $20\times 20$-lattice as a function of the number of Monte Carlo cycles\label{b_chi}, starting with random spins at $T=2.4$}
\end{figure}

\begin{figure}[h]
	\includegraphics[width=\textwidth]{b1.png}
	\caption{The number of accepted spin flips as function of the number of cycles at $T=1.0$}
\end{figure}
\begin{figure}[h]
	\includegraphics[width=\textwidth]{b24.png}
	\caption{The number of accepted spin flips as function of the number of cycles at $T=2.4$}
\end{figure}

In the next step, we took a closer look at the probability of the single values of $E$ to appear. We observed a $2\times2$-lattice at a temperature of $T=1.0$ respectively $T=2.4$ again and did two histograms of how often an energy value occurred. Obviously, that distribution is expanded for a higher temperature. This correspondents with the fact that the variation of the energy for these temperatures is higher, which means that more values are accessed. Both histograms can be found in fig \ref{d1} -- \ref{d2}.

\begin{figure}[h]
	\includegraphics[width=\textwidth]{d1.png}
	\caption{Frequency of occurrence for different energies at $T=1.0$}
\end{figure}
\begin{figure}[h]
	\includegraphics[width=\textwidth]{d24.png}
	\caption{Frequency of occurrence for different energies at $T=2.4$}
\end{figure}
\section{Comparison and discussion of results}

\section{source code}

\end{document}
