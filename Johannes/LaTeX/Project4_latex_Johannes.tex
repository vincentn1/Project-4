\documentclass[10pt,a4paper]{article}
\usepackage[utf8]{inputenc}
\usepackage[english]{babel}
\usepackage[T1]{fontenc}
\usepackage{amsmath}
\usepackage{amsfonts}
\usepackage{amssymb}
\usepackage{makeidx}
\usepackage{graphicx}
\usepackage{fourier}
\usepackage{listings}
\usepackage{color}
\usepackage{hyperref}
\usepackage[left=2cm,right=2cm,top=2cm,bottom=2cm]{geometry}
\author{Johannes Scheller, Vincent Noculak, Lukas Powalla, Richard Asbah}
\title{Computational Physics - Project 4}

\lstset{language=C++,
	keywordstyle=\bfseries\color{blue},
	commentstyle=\itshape\color{red},
	stringstyle=\color{green},
	identifierstyle=\bfseries,
	frame=single}
\begin{document}

\maketitle
\newpage
\tableofcontents
\newpage

\section*{Introduction}

\section{Theory}

\section{Execution}
\subsection{Implementing the Algorithm}
\subsection{Results}
As a first benchmark test for our program, we calculated the expectation values $<E>$, $<|M>|$, $<C_V>$ and $<\chi>$ of a $2\times2$-lattice for different temperatures. Those results could be easily compared to the analytical values from the part \ref{closed_solution}. In fig. \ref{b_E} - \ref{b_chi}, you can see our results for these expectation values compared with the analytical solutions as functions of $T$. We only needed about $NNNN$ Monte Carlo cycles to get achieve good results for this lattice size!
\section{Comparison and discussion of results}

\section{source code}

\end{document}
